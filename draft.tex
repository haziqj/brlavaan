% Options for packages loaded elsewhere
\PassOptionsToPackage{unicode}{hyperref}
\PassOptionsToPackage{hyphens}{url}
%
\documentclass[
]{article}
\usepackage{amsmath,amssymb}
\usepackage{iftex}
\ifPDFTeX
  \usepackage[T1]{fontenc}
  \usepackage[utf8]{inputenc}
  \usepackage{textcomp} % provide euro and other symbols
\else % if luatex or xetex
  \usepackage{unicode-math} % this also loads fontspec
  \defaultfontfeatures{Scale=MatchLowercase}
  \defaultfontfeatures[\rmfamily]{Ligatures=TeX,Scale=1}
\fi
\usepackage{lmodern}
\ifPDFTeX\else
  % xetex/luatex font selection
\fi
% Use upquote if available, for straight quotes in verbatim environments
\IfFileExists{upquote.sty}{\usepackage{upquote}}{}
\IfFileExists{microtype.sty}{% use microtype if available
  \usepackage[]{microtype}
  \UseMicrotypeSet[protrusion]{basicmath} % disable protrusion for tt fonts
}{}
\makeatletter
\@ifundefined{KOMAClassName}{% if non-KOMA class
  \IfFileExists{parskip.sty}{%
    \usepackage{parskip}
  }{% else
    \setlength{\parindent}{0pt}
    \setlength{\parskip}{6pt plus 2pt minus 1pt}}
}{% if KOMA class
  \KOMAoptions{parskip=half}}
\makeatother
\usepackage{xcolor}
\usepackage[margin=1in]{geometry}
\usepackage{color}
\usepackage{fancyvrb}
\newcommand{\VerbBar}{|}
\newcommand{\VERB}{\Verb[commandchars=\\\{\}]}
\DefineVerbatimEnvironment{Highlighting}{Verbatim}{commandchars=\\\{\}}
% Add ',fontsize=\small' for more characters per line
\usepackage{framed}
\definecolor{shadecolor}{RGB}{248,248,248}
\newenvironment{Shaded}{\begin{snugshade}}{\end{snugshade}}
\newcommand{\AlertTok}[1]{\textcolor[rgb]{0.94,0.16,0.16}{#1}}
\newcommand{\AnnotationTok}[1]{\textcolor[rgb]{0.56,0.35,0.01}{\textbf{\textit{#1}}}}
\newcommand{\AttributeTok}[1]{\textcolor[rgb]{0.13,0.29,0.53}{#1}}
\newcommand{\BaseNTok}[1]{\textcolor[rgb]{0.00,0.00,0.81}{#1}}
\newcommand{\BuiltInTok}[1]{#1}
\newcommand{\CharTok}[1]{\textcolor[rgb]{0.31,0.60,0.02}{#1}}
\newcommand{\CommentTok}[1]{\textcolor[rgb]{0.56,0.35,0.01}{\textit{#1}}}
\newcommand{\CommentVarTok}[1]{\textcolor[rgb]{0.56,0.35,0.01}{\textbf{\textit{#1}}}}
\newcommand{\ConstantTok}[1]{\textcolor[rgb]{0.56,0.35,0.01}{#1}}
\newcommand{\ControlFlowTok}[1]{\textcolor[rgb]{0.13,0.29,0.53}{\textbf{#1}}}
\newcommand{\DataTypeTok}[1]{\textcolor[rgb]{0.13,0.29,0.53}{#1}}
\newcommand{\DecValTok}[1]{\textcolor[rgb]{0.00,0.00,0.81}{#1}}
\newcommand{\DocumentationTok}[1]{\textcolor[rgb]{0.56,0.35,0.01}{\textbf{\textit{#1}}}}
\newcommand{\ErrorTok}[1]{\textcolor[rgb]{0.64,0.00,0.00}{\textbf{#1}}}
\newcommand{\ExtensionTok}[1]{#1}
\newcommand{\FloatTok}[1]{\textcolor[rgb]{0.00,0.00,0.81}{#1}}
\newcommand{\FunctionTok}[1]{\textcolor[rgb]{0.13,0.29,0.53}{\textbf{#1}}}
\newcommand{\ImportTok}[1]{#1}
\newcommand{\InformationTok}[1]{\textcolor[rgb]{0.56,0.35,0.01}{\textbf{\textit{#1}}}}
\newcommand{\KeywordTok}[1]{\textcolor[rgb]{0.13,0.29,0.53}{\textbf{#1}}}
\newcommand{\NormalTok}[1]{#1}
\newcommand{\OperatorTok}[1]{\textcolor[rgb]{0.81,0.36,0.00}{\textbf{#1}}}
\newcommand{\OtherTok}[1]{\textcolor[rgb]{0.56,0.35,0.01}{#1}}
\newcommand{\PreprocessorTok}[1]{\textcolor[rgb]{0.56,0.35,0.01}{\textit{#1}}}
\newcommand{\RegionMarkerTok}[1]{#1}
\newcommand{\SpecialCharTok}[1]{\textcolor[rgb]{0.81,0.36,0.00}{\textbf{#1}}}
\newcommand{\SpecialStringTok}[1]{\textcolor[rgb]{0.31,0.60,0.02}{#1}}
\newcommand{\StringTok}[1]{\textcolor[rgb]{0.31,0.60,0.02}{#1}}
\newcommand{\VariableTok}[1]{\textcolor[rgb]{0.00,0.00,0.00}{#1}}
\newcommand{\VerbatimStringTok}[1]{\textcolor[rgb]{0.31,0.60,0.02}{#1}}
\newcommand{\WarningTok}[1]{\textcolor[rgb]{0.56,0.35,0.01}{\textbf{\textit{#1}}}}
\usepackage{graphicx}
\makeatletter
\def\maxwidth{\ifdim\Gin@nat@width>\linewidth\linewidth\else\Gin@nat@width\fi}
\def\maxheight{\ifdim\Gin@nat@height>\textheight\textheight\else\Gin@nat@height\fi}
\makeatother
% Scale images if necessary, so that they will not overflow the page
% margins by default, and it is still possible to overwrite the defaults
% using explicit options in \includegraphics[width, height, ...]{}
\setkeys{Gin}{width=\maxwidth,height=\maxheight,keepaspectratio}
% Set default figure placement to htbp
\makeatletter
\def\fps@figure{htbp}
\makeatother
\setlength{\emergencystretch}{3em} % prevent overfull lines
\providecommand{\tightlist}{%
  \setlength{\itemsep}{0pt}\setlength{\parskip}{0pt}}
\setcounter{secnumdepth}{-\maxdimen} % remove section numbering
\newlength{\cslhangindent}
\setlength{\cslhangindent}{1.5em}
\newlength{\csllabelwidth}
\setlength{\csllabelwidth}{3em}
\newlength{\cslentryspacingunit} % times entry-spacing
\setlength{\cslentryspacingunit}{\parskip}
\newenvironment{CSLReferences}[2] % #1 hanging-ident, #2 entry spacing
 {% don't indent paragraphs
  \setlength{\parindent}{0pt}
  % turn on hanging indent if param 1 is 1
  \ifodd #1
  \let\oldpar\par
  \def\par{\hangindent=\cslhangindent\oldpar}
  \fi
  % set entry spacing
  \setlength{\parskip}{#2\cslentryspacingunit}
 }%
 {}
\usepackage{calc}
\newcommand{\CSLBlock}[1]{#1\hfill\break}
\newcommand{\CSLLeftMargin}[1]{\parbox[t]{\csllabelwidth}{#1}}
\newcommand{\CSLRightInline}[1]{\parbox[t]{\linewidth - \csllabelwidth}{#1}\break}
\newcommand{\CSLIndent}[1]{\hspace{\cslhangindent}#1}
\newcommand{\bzero}{{\mathbf 0}}
\newcommand{\bone}{{\mathbf 1}}
\newcommand{\ba}{{\mathbf a}}
\newcommand{\bb}{{\mathbf b}}
\newcommand{\bc}{{\mathbf c}}
\newcommand{\bd}{{\mathbf d}}
\newcommand{\be}{{\mathbf e}}
\newcommand{\bff}{{\mathbf f}}
\newcommand{\bg}{{\mathbf g}}
\newcommand{\bh}{{\mathbf h}}
\newcommand{\bi}{{\mathbf i}}
\newcommand{\bj}{{\mathbf j}}
\newcommand{\bk}{{\mathbf k}}
\newcommand{\bl}{{\mathbf l}}
\newcommand{\bmm}{{\mathbf m}}
\newcommand{\bn}{{\mathbf n}}
\newcommand{\bo}{{\mathbf o}}
\newcommand{\bp}{{\mathbf p}}
\newcommand{\bq}{{\mathbf q}}
\newcommand{\br}{{\mathbf r}}
\newcommand{\bs}{{\mathbf s}}
\newcommand{\bt}{{\mathbf t}}
\newcommand{\bu}{{\mathbf u}}
\newcommand{\bv}{{\mathbf v}}
\newcommand{\bw}{{\mathbf w}}
\newcommand{\bx}{{\mathbf x}}
\newcommand{\by}{{\mathbf y}}
\newcommand{\bz}{{\mathbf z}}
\newcommand{\bA}{{\mathbf A}}
\newcommand{\bB}{{\mathbf B}}
\newcommand{\bC}{{\mathbf C}}
\newcommand{\bD}{{\mathbf D}}
\newcommand{\bE}{{\mathbf E}}
\newcommand{\bF}{{\mathbf F}}
\newcommand{\bG}{{\mathbf G}}
\newcommand{\bH}{{\mathbf H}}
\newcommand{\bI}{{\mathbf I}}
\newcommand{\bJ}{{\mathbf J}}
\newcommand{\bK}{{\mathbf K}}
\newcommand{\bL}{{\mathbf L}}
\newcommand{\bM}{{\mathbf M}}
\newcommand{\bN}{{\mathbf N}}
\newcommand{\bO}{{\mathbf O}}
\newcommand{\bP}{{\mathbf P}}
\newcommand{\bQ}{{\mathbf Q}}
\newcommand{\bR}{{\mathbf R}}
\newcommand{\bS}{{\mathbf S}}
\newcommand{\bT}{{\mathbf T}}
\newcommand{\bU}{{\mathbf U}}
\newcommand{\bV}{{\mathbf V}}
\newcommand{\bW}{{\mathbf W}}
\newcommand{\bX}{{\mathbf X}}
\newcommand{\bY}{{\mathbf Y}}
\newcommand{\bZ}{{\mathbf Z}}
\newcommand{\balpha}{{\boldsymbol\alpha}}
\newcommand{\bbeta}{{\boldsymbol\beta}}
\newcommand{\bgamma}{{\boldsymbol\gamma}}
\newcommand{\bdelta}{{\boldsymbol\delta}}
\newcommand{\bepsilon}{{\boldsymbol\epsilon}}
\newcommand{\bvarepsilon}{{\boldsymbol\varepsilon}}
\newcommand{\bzeta}{{\boldsymbol\zeta}}
\newcommand{\bfeta}{{\boldsymbol\eta}}
\newcommand{\boldeta}{{\boldsymbol\eta}}
\newcommand{\btheta}{{\boldsymbol\theta}}
\newcommand{\bvartheta}{{\boldsymbol\vartheta}}
\newcommand{\biota}{{\boldsymbol\iota}}
\newcommand{\bkappa}{{\boldsymbol\kappa}}
\newcommand{\blambda}{{\boldsymbol\lambda}}
\newcommand{\bmu}{{\boldsymbol\mu}}
\newcommand{\bnu}{{\boldsymbol\nu}}
\newcommand{\bxi}{{\boldsymbol\xi}}
\newcommand{\bpi}{{\boldsymbol\pi}}
\newcommand{\bvarpi}{{\boldsymbol\varpi}}
\newcommand{\brho}{{\boldsymbol\rho}}
\newcommand{\bvarrho}{{\boldsymbol\varrho}}
\newcommand{\bsigma}{{\boldsymbol\sigma}}
\newcommand{\bvarsigma}{{\boldsymbol\varsigma}}
\newcommand{\btau}{{\boldsymbol\tau}}
\newcommand{\bupsilon}{{\boldsymbol\upsilon}}
\newcommand{\bphi}{{\boldsymbol\phi}}
\newcommand{\bvarphi}{{\boldsymbol\varphi}}
\newcommand{\bchi}{{\boldsymbol\chi}}
\newcommand{\bpsi}{{\boldsymbol\psi}}
\newcommand{\bomega}{{\boldsymbol\omega}}
\newcommand{\bGamma}{{\boldsymbol\Gamma}}
\newcommand{\bDelta}{{\boldsymbol\Delta}}
\newcommand{\bTheta}{{\boldsymbol\Theta}}
\newcommand{\bLambda}{{\boldsymbol\Lambda}}
\newcommand{\bXi}{{\boldsymbol\Xi}}
\newcommand{\bPi}{{\boldsymbol\Pi}}
\newcommand{\bSigma}{{\boldsymbol\Sigma}}
\newcommand{\bUpsilon}{{\boldsymbol\Upsilon}}
\newcommand{\bPhi}{{\boldsymbol\Phi}}
\newcommand{\bPsi}{{\boldsymbol\Psi}}
\newcommand{\bOmega}{{\boldsymbol\Omega}}
\DeclareMathOperator{\diag}{diag}
\DeclareMathOperator{\Prob}{P}
\DeclareMathOperator{\E}{E}
\DeclareMathOperator{\Var}{Var}
\DeclareMathOperator{\Cov}{Cov}
\DeclareMathOperator{\Corr}{Corr}
\DeclareMathOperator{\sd}{sd}
\DeclareMathOperator{\se}{se}
\DeclareMathOperator{\N}{N}
\DeclareMathOperator{\Bin}{Bin}
\DeclareMathOperator{\Bern}{Bern}
\DeclareMathOperator{\Dir}{Dir}
\DeclareMathOperator{\Wis}{Wis}
\DeclareMathOperator{\logit}{logit}
\DeclareMathOperator{\expit}{expit}
\DeclareMathOperator{\Mult}{Mult}
\DeclareMathOperator{\Cat}{Cat}
\DeclareMathOperator{\Pois}{Poi}
\DeclareMathOperator{\Geom}{Geom}
\DeclareMathOperator{\NBin}{NBin}
\DeclareMathOperator{\Exp}{Exp}
\DeclareMathOperator{\Betadist}{Beta}
\DeclareMathOperator{\Hypergeom}{Hypergeom}
\DeclareMathOperator{\Cauchy}{Cauchy}
\DeclareMathOperator{\hCauchy}{half-Cauchy}
\DeclareMathOperator{\LKJ}{LKJ}
\DeclareMathOperator{\Unif}{Unif}
\DeclareMathOperator{\KL}{KL}
\DeclareMathOperator{\ind}{\mathds{1}}
\newcommand{\iid}{\,\overset{\text{iid}}{\sim}\,}
\DeclareMathOperator*{\plim}{plim}
\DeclareMathOperator{\Lik}{L}
\DeclareMathOperator{\argmax}{argmax}

\newcommand{\bbR}{\mathbb{R}}
\newcommand{\bbN}{\mathbb{N}}
\newcommand{\bbZ}{\mathbb{Z}}
\newcommand{\bbC}{\mathbb{C}}
\newcommand{\bbS}{\mathbb{S}}
\newcommand{\bbH}{\mathbb{H}}
\newcommand{\bbP}{\mathbb{P}}

\newcommand{\cA}{{\mathcal A}}
\newcommand{\cB}{{\mathcal B}}
\newcommand{\cC}{{\mathcal C}}
\newcommand{\cD}{{\mathcal D}}
\newcommand{\cE}{{\mathcal E}}
\newcommand{\cF}{{\mathcal F}}
\newcommand{\cG}{{\mathcal G}}
\newcommand{\cH}{{\mathcal H}}
\newcommand{\cI}{{\mathcal I}}
\newcommand{\cJ}{{\mathcal J}}
\newcommand{\cK}{{\mathcal K}}
\newcommand{\cL}{{\mathcal L}}
\newcommand{\cM}{{\mathcal M}}
\newcommand{\cN}{{\mathcal N}}
\newcommand{\cO}{{\mathcal O}}
\newcommand{\cP}{{\mathcal P}}
\newcommand{\cQ}{{\mathcal Q}}
\newcommand{\cR}{{\mathcal R}}
\newcommand{\cS}{{\mathcal S}}
\newcommand{\cT}{{\mathcal T}}
\newcommand{\cU}{{\mathcal U}}
\newcommand{\cV}{{\mathcal V}}
\newcommand{\cW}{{\mathcal W}}
\newcommand{\cX}{{\mathcal X}}
\newcommand{\cY}{{\mathcal Y}}
\newcommand{\cZ}{{\mathcal Z}}

\newcommand{\myoverbrace}[3][gray]{{\color{#1}\overbrace{\color{black}#2}^{#3}}}
\newcommand{\myunderbrace}[3][gray]{{\color{#1}\underbrace{\color{black}#2}_{#3}}}

\usepackage{booktabs}
\usepackage{longtable}
\usepackage{array}
\usepackage{multirow}
\usepackage{wrapfig}
\usepackage{float}
\usepackage{colortbl}
\usepackage{pdflscape}
\usepackage{tabu}
\usepackage{threeparttable}
\usepackage{threeparttablex}
\usepackage[normalem]{ulem}
\usepackage{makecell}
\usepackage{xcolor}
\ifLuaTeX
  \usepackage{selnolig}  % disable illegal ligatures
\fi
\IfFileExists{bookmark.sty}{\usepackage{bookmark}}{\usepackage{hyperref}}
\IfFileExists{xurl.sty}{\usepackage{xurl}}{} % add URL line breaks if available
\urlstyle{same}
\hypersetup{
  pdftitle={Bias Reduction PML},
  hidelinks,
  pdfcreator={LaTeX via pandoc}}

\title{Bias Reduction PML}
\author{}
\date{\vspace{-2.5em}2023-11-20}

\begin{document}
\maketitle

\newcommand{\pimod}[1]{\pi_{#1}(\btheta)}
\newcommand{\Sigmaystar}{\bSigma_{\by^*}}
\newcommand{\pl}{\operatorname{\ell_P}}
\newcommand{\mlepl}{\hat\btheta_{\text{PL}}}
\newcommand{\mle}{\hat\btheta_{\text{ML}}}
\newcommand{\pimodpl}{\pi_{y_iy_j}^{(ij)}(\btheta)}
\newcommand{\tr}{\operatorname{tr}}

\hypertarget{introduction}{%
\section{Introduction}\label{introduction}}

Let \(\bY = (Y_1, \dots, Y_p)^\top \in \{0,1\}^p\) be a vector of
Bernoulli random variables. Consider a response pattern
\(\by = (y_1,\dots,y_p)^\top\), where each \(y_i\in\{0,1\}\). The
probability of observing such a response pattern is given by the joint
distribution \begin{align}\label{eq:jointprob1}
\pi
&= \Pr(\bY = \by)  = \Pr(Y_1=y_1,\dots,Y_p=y_p).
%(\#eq:jointprob1)
\end{align} Note that there are a total of \(R=2^p\) possible joint
probabilities \(\pi_r\) corresponding to all possible two-way response
patterns \(\by_r\).

When we consider a parametric model with parameter vector
\(\btheta\in\bbR^m\), we write \(\pi_r(\btheta)\) to indicate each joint
probability, and \begin{equation}\label{eq:jointprob2}
\bpi(\btheta) = \begin{pmatrix}
\pi_{1}(\btheta) \\
\vdots \\
\pi_{R}(\btheta)  \\
\end{pmatrix} \in [0,1]^R
%(\#eq:jointprob2)
\end{equation} for the vector of joint probabilities, with
\(\sum_{r=1}^R \pi_{r}(\btheta) =1\).

\hypertarget{binary-factor-models}{%
\section{Binary factor models}\label{binary-factor-models}}

The model of interest is a factor model, commonly used in social
statistics. Using an underlying variable (UV) approach, the observed
binary responses \(y_i\) are manifestations of some latent, continuous
variables \(Y_i^*\), \(i=1,\dots,p\). The connection is made as follows:
\[
Y_i = \begin{cases}
1 & Y_i^* > \tau_i \\
0 & Y_i^* \leq \tau_i,
\end{cases}
\] where \(\tau_i\) is the threshold associated with the variable
\(Y_i^*\). For convenience, \(Y_i^*\) is taken to be standard normal
random variables\footnote{For parameter identifiability, the location
  and scale of the normal distribution have to be fixed if the
  thresholds are to be estimated.}. The factor model takes the form \[
\mathbf Y^* = \mathbf \Lambda\boldsymbol \eta + \boldsymbol \epsilon,
\] where each component is explained below:

\begin{itemize}
\item
  \(\mathbf Y^* = (Y_1^*,\dots,Y_p^*)^\top \in \mathbf R^p\) are the
  underlying variables;
\item
  \(\boldsymbol\Lambda \in \mathbf R^{p \times q}\) is the matrix of
  loadings;
\item
  \(\boldsymbol \eta = (\eta_1,\dots,\eta_q)^\top \in \mathbf R^q\) is
  the vector of latent factors;
\item
  \(\boldsymbol \epsilon \in \mathbf R^p\) are the error terms
  associated with the items (aka unique variables).
\end{itemize}

We also make some distributional assumptions, namely

\begin{enumerate}
\def\labelenumi{\arabic{enumi}.}
\item
  \(\boldsymbol\eta \sim \operatorname{N}_q(\mathbf 0, \boldsymbol\Psi)\),
  where \(\bPsi\) is a correlation matrix, i.e.~for
  \(k,l\in\{1,\dots,q\}\), \[
  \bPsi_{kl} = \begin{cases}
  1 & \text{if }  k = l \\
  \rho(\eta_k, \eta_l) & \text{if } k \neq l.
  \end{cases}
  \]
\item
  \(\boldsymbol \epsilon \sim \operatorname{N}_p(\mathbf 0, \boldsymbol\Theta_\epsilon)\),
  with
  \(\boldsymbol\Theta_\epsilon = \mathbf I - \operatorname{diag}(\boldsymbol\Lambda \boldsymbol\Psi \boldsymbol\Lambda^\top)\).
\end{enumerate}

These two assumptions, together with
\(\operatorname{Cov}(\boldsymbol\eta, \boldsymbol\epsilon) = 0\),
implies that
\(\mathbf Y^*\sim\operatorname{N}_p(\mathbf 0,\bSigma_{\by^*})\), where
\begin{align}
\bSigma_{\by^*}
= \operatorname{Var}(\mathbf Y^*) 
&= \boldsymbol\Lambda\boldsymbol\Phi\boldsymbol\Lambda^\top + \boldsymbol\Theta_\epsilon.
%&= \mathbf I + (\boldsymbol\Lambda\boldsymbol\Phi\boldsymbol\Lambda^\top - \operatorname{diag}\big(\boldsymbol\Lambda \boldsymbol\Phi \boldsymbol\Lambda^\top)\big).
\end{align} The parameter vector for this factor model is denoted
\(\boldsymbol\theta^\top = (\boldsymbol\lambda, \boldsymbol\psi, \boldsymbol\tau) \in\bbR^m\),
where it contains the vectors of the free non-redundant parameters in
\(\boldsymbol\Lambda\) and \(\boldsymbol \Psi\) respectively, as well as
the vector of all free thresholds.

Under this factor model, the probability of response pattern
\(\mathbf y_r\) is \begin{align}
\pi_{r}(\btheta)
&= \Pr(\mathbf Y = \mathbf y_r \mid \boldsymbol\theta) \\
&= \idotsint_A \phi_p(\mathbf y^* \mid \mathbf 0, \boldsymbol\Sigma_{\mathbf y^*} ) \, \text{d}\mathbf y^*
\end{align} where
\(\phi_p(\cdot \mid \boldsymbol\mu,\boldsymbol\Sigma)\) is the density
function of the \(p\)-dimensional normal distribution with mean
\(\boldsymbol\mu\) and variance \(\boldsymbol\Sigma\). This integral is
evaluated on the set \[
A = \{ \mathbf Y^* \in \mathbb R^p \mid Y_1=y_1,\dots,Y_p=y_p \}.
\]

\hypertarget{pairwise-likelihood-estimation}{%
\section{Pairwise likelihood
estimation}\label{pairwise-likelihood-estimation}}

In order to define the pairwise likelihood, let
\(\pi_{y_iy_j}^{(ij)}(\btheta)\) be the probability under the model that
\(Y_i=y_i \in \{0,1\}\) and \(Y_j=y_j\in\{0,1\}\) for a pair of
variables \(Y_i\) and \(Y_j\), \(i,j=1,\dots,p\) and \(i<j\). The
pairwise log-likelihood takes the form \begin{equation}
\operatorname{\ell_P}(\btheta) = \sum_{i<j} \sum_{y_i}\sum_{y_j} \hat n_{y_iy_j}^{(ij)} \log \pi_{y_iy_j}^{(ij)}(\btheta),
\end{equation} where \(\hat n_{y_iy_j}^{(ij)}\) is the observed
(weighted) frequency of sample units with \(Y_i=y_i\) and \(Y_j=y_j\),
\[
\hat n_{y_iy_j}^{(ij)} = \sum_h w_h [\by^{(h)}_i = y_i, \by^{(h)}_j = y_j].
\] Here the \(w_h\) refers to the design weight for any individual \(h\)
in the sample. For simplicity, we may assume that these weights are
normalised such that \(\sum w_h = N\). In such a case, a simple random
sampling design would imply all weights are equal to one, and the
weighted pairwise likelihood reduces to the usual pairwise likelihood
function.

Let us also define the corresponding observed pairwise proportions
\(p_{y_iy_j}^{(ij)} = \hat n_{y_iy_j}^{(ij)}/n\). There are a total of
\(\tilde R = 4 \times {p \choose 2}\) summands, where the `4' is
representative of the total number of pairwise combinations of binary
choices `00', `10', `01', and `11'.

The \emph{pairwise maximum likelihood estimator}
\(\hat\btheta_{\text{PL}}\) satisfies
\(\hat\btheta_{\text{PL}}= \argmax_{\btheta} \operatorname{\ell_P}(\btheta)\).
Under certain regularity conditions, \begin{equation}
\sqrt N (\hat\btheta_{\text{PL}}- \btheta) \xrightarrow{D} {\N}_m\big(\bzero, \cH(\btheta)\cJ(\btheta)^{-1}\cH(\btheta)\big),
\end{equation} where

\begin{itemize}
\tightlist
\item
  \(\cH(\btheta)=-\E\nabla^2\operatorname{\ell_P}(\btheta;\by^{(h)})\)
  is the \emph{sensitivity matrix}; and
\item
  \(\cJ(\btheta)=\Var\big( \nabla\operatorname{\ell_P}(\btheta;\by^{(h)})\big)\)
  is the \emph{variability matrix}.
\end{itemize}

In practice, we may estimate these matrices using the following
estimators:

\[
\hat\bH := \bH(\hat\btheta)  = -\frac{1}{\sum_h w_h}\sum_{h=1}^N w_h\nabla^2\operatorname{\ell_P}(\btheta;\by^{(h)}) \bigg|_{\btheta = \hat\btheta}
\]

\[
\hat\bJ := \bJ(\hat\btheta) = \frac{1}{\sum_h w_h}\sum_{h=1}^N w_h^2\nabla\operatorname{\ell_P}(\btheta;\by^{(h)})\nabla\operatorname{\ell_P}(\btheta;\by^{(h)})^\top \bigg|_{\btheta = \hat\btheta}
\]

That is, \(\hat\bH\) is the Hessian resulting from the optimisation of
the pairwise likelihood function, while \(\hat\bJ\) is the cross product
of the gradient of the pairwise likelihood function--each evaluated at
the maximum PLE. Note that both are considered ``unit'' information
matrices, as they are normalised by the sum of the weights (sample
size).

\hypertarget{bias-reduction}{%
\section{Bias reduction}\label{bias-reduction}}

Define \[
A(\hat\btheta) = - \frac{1}{2} \nabla \operatorname{tr}\left( \bH(\btheta)^{-1}\bJ(\btheta)^{-1} \right) \bigg|_{\btheta = \hat\btheta} .
\] Then, an improved estimator \(\tilde\btheta\) is given by \[
\tilde\btheta = \hat\btheta + \bH(\hat\btheta)^{-1}A(\hat\btheta).
\]

Some computational notes:

\begin{itemize}
\tightlist
\item
  The Hessian \(\bH(\btheta)\) matrix is obtained as a byproduct of the
  optimisation routine in \texttt{\{lavaan\}} (or using my manually
  coded pml function and optim). There is no explicit code for it.
\item
  The variability matrix \(\bJ(\btheta)\) is obtained from
  \texttt{\{lavaan\}}, by tricking it into accepting starting values
  \(\btheta\) as converged parameter values and extracting the
  \(\bJ(\btheta)\) accordingly.
\item
  Then form the \(\bA(\btheta)\) matrix and obtain the gradient using
  the \texttt{numDeriv} package.
\end{itemize}

\hypertarget{example}{%
\section{Example}\label{example}}

Consider \(p=5\) binary items generated using the UV approach as above
with the following true parameter values:

\begin{itemize}
\tightlist
\item
  \(\blambda\) = 0.8, 0.7, 0.47, 0.38, 0.34
\item
  \(\btau\) = -1.43, -0.55, -0.13, -0.72, -1.13
\end{itemize}

NB: This is called Model no 1 from a previous work (Jamil, Moustaki, and
Skinner 2023).

\begin{Shaded}
\begin{Highlighting}[]
\FunctionTok{set.seed}\NormalTok{(}\DecValTok{123}\NormalTok{)}
\NormalTok{model\_no }\OtherTok{\textless{}{-}} \DecValTok{1}
\NormalTok{mod }\OtherTok{\textless{}{-}} \FunctionTok{txt\_mod}\NormalTok{(model\_no)}
\NormalTok{(dat }\OtherTok{\textless{}{-}} \FunctionTok{gen\_data\_bin}\NormalTok{(model\_no))}
\end{Highlighting}
\end{Shaded}

\begin{verbatim}
## # A tibble: 1,000 x 5
##    y1    y2    y3    y4    y5   
##    <ord> <ord> <ord> <ord> <ord>
##  1 1     1     0     1     1    
##  2 1     1     1     1     1    
##  3 1     0     1     0     1    
##  4 1     0     1     1     1    
##  5 1     0     1     0     1    
##  6 1     0     0     1     0    
##  7 1     1     1     0     1    
##  8 1     0     0     1     1    
##  9 1     0     1     1     1    
## 10 1     1     1     0     1    
## # i 990 more rows
\end{verbatim}

The fit from the PL routine is obtained from \texttt{\{lavaan\}}:

\begin{Shaded}
\begin{Highlighting}[]
\NormalTok{fit\_lav }\OtherTok{\textless{}{-}} \FunctionTok{sem}\NormalTok{(mod, dat, }\AttributeTok{std.lv =} \ConstantTok{TRUE}\NormalTok{, }\AttributeTok{estimator =} \StringTok{"PML"}\NormalTok{)}
\FunctionTok{summary}\NormalTok{(fit\_lav)}
\end{Highlighting}
\end{Shaded}

\begin{verbatim}
## lavaan 0.6.17.1946 ended normally after 17 iterations
## 
##   Estimator                                        PML
##   Optimization method                           NLMINB
##   Number of model parameters                        10
## 
##   Number of observations                          1000
## 
## Model Test User Model:
##                                               Standard      Scaled
##   Test Statistic                                 2.478       3.571
##   Degrees of freedom                                 5       5.320
##   P-value (Unknown)                                 NA       0.655
##   Scaling correction factor                                  0.694
##     mean+var adjusted correction (PML)                            
## 
## Parameter Estimates:
## 
##   Standard errors                             Sandwich
##   Information bread                           Observed
##   Observed information based on                Hessian
## 
## Latent Variables:
##                    Estimate  Std.Err  z-value  P(>|z|)
##   eta1 =~                                             
##     y1                0.696    0.077    9.092    0.000
##     y2                0.700    0.062   11.302    0.000
##     y3                0.447    0.060    7.496    0.000
##     y4                0.493    0.061    8.138    0.000
##     y5                0.417    0.072    5.798    0.000
## 
## Intercepts:
##                    Estimate  Std.Err  z-value  P(>|z|)
##    .y1                0.000                           
##    .y2                0.000                           
##    .y3                0.000                           
##    .y4                0.000                           
##    .y5                0.000                           
##     eta1              0.000                           
## 
## Thresholds:
##                    Estimate  Std.Err  z-value  P(>|z|)
##     y1|t1            -1.522    0.062  -24.640    0.000
##     y2|t1            -0.550    0.042  -13.135    0.000
##     y3|t1            -0.171    0.040   -4.297    0.000
##     y4|t1            -0.678    0.043  -15.714    0.000
##     y5|t1            -1.190    0.052  -23.011    0.000
## 
## Variances:
##                    Estimate  Std.Err  z-value  P(>|z|)
##    .y1                0.516                           
##    .y2                0.510                           
##    .y3                0.800                           
##    .y4                0.757                           
##    .y5                0.826                           
##     eta1              1.000                           
## 
## Scales y*:
##                    Estimate  Std.Err  z-value  P(>|z|)
##     y1                1.000                           
##     y2                1.000                           
##     y3                1.000                           
##     y4                1.000                           
##     y5                1.000
\end{verbatim}

We compare the fit of the coefficients and the true value:

\begin{Shaded}
\begin{Highlighting}[]
\NormalTok{theta\_hat }\OtherTok{\textless{}{-}} \FunctionTok{coef}\NormalTok{(fit\_lav)}
\NormalTok{theta\_true }\OtherTok{\textless{}{-}} \FunctionTok{c}\NormalTok{(lavaan.bingof}\SpecialCharTok{:::}\FunctionTok{get\_Lambda}\NormalTok{(model\_no),}
\NormalTok{                lavaan.bingof}\SpecialCharTok{:::}\FunctionTok{get\_tau}\NormalTok{(model\_no))}
\FunctionTok{tibble}\NormalTok{(}
  \AttributeTok{coef =} \FunctionTok{names}\NormalTok{(theta\_hat),}
  \AttributeTok{theta\_hat =} \FunctionTok{round}\NormalTok{(theta\_hat, }\DecValTok{2}\NormalTok{),}
  \AttributeTok{theta\_true =}\NormalTok{ theta\_true}
\NormalTok{) }\SpecialCharTok{|\textgreater{}}
  \FunctionTok{kbl}\NormalTok{(}\AttributeTok{booktabs =} \ConstantTok{TRUE}\NormalTok{)}
\end{Highlighting}
\end{Shaded}

\begin{tabular}[t]{lrr}
\toprule
coef & theta\_hat & theta\_true\\
\midrule
eta1=\textasciitilde{}y1 & 0.70 & 0.80\\
eta1=\textasciitilde{}y2 & 0.70 & 0.70\\
eta1=\textasciitilde{}y3 & 0.45 & 0.47\\
eta1=\textasciitilde{}y4 & 0.49 & 0.38\\
eta1=\textasciitilde{}y5 & 0.42 & 0.34\\
\addlinespace
y1|t1 & -1.52 & -1.43\\
y2|t1 & -0.55 & -0.55\\
y3|t1 & -0.17 & -0.13\\
y4|t1 & -0.68 & -0.72\\
y5|t1 & -1.19 & -1.13\\
\bottomrule
\end{tabular}

Now apply the bias reduction method

\begin{Shaded}
\begin{Highlighting}[]
\CommentTok{\# Assume HHH() and JJJ() are the functions to obtain the H and J matrices at a}
\CommentTok{\# given theta}
\NormalTok{AAA }\OtherTok{\textless{}{-}} \ControlFlowTok{function}\NormalTok{(.theta) \{}
\NormalTok{  tmp }\OtherTok{\textless{}{-}} \ControlFlowTok{function}\NormalTok{(x) \{}
\NormalTok{    Hinv }\OtherTok{\textless{}{-}} \FunctionTok{HHH}\NormalTok{(x) }\SpecialCharTok{|\textgreater{}}\NormalTok{ MASS}\SpecialCharTok{::}\FunctionTok{ginv}\NormalTok{()}
\NormalTok{    J    }\OtherTok{\textless{}{-}} \FunctionTok{JJJ}\NormalTok{(x)}
    \SpecialCharTok{{-}}\FloatTok{0.5} \SpecialCharTok{*} \FunctionTok{sum}\NormalTok{(}\FunctionTok{diag}\NormalTok{(Hinv }\SpecialCharTok{\%*\%}\NormalTok{ J))}
\NormalTok{  \}}
\NormalTok{  numDeriv}\SpecialCharTok{::}\FunctionTok{grad}\NormalTok{(tmp, .theta)}
\NormalTok{\}}

\CommentTok{\# Bias reduction}
\NormalTok{(A }\OtherTok{\textless{}{-}} \FunctionTok{AAA}\NormalTok{(theta\_hat))}
\end{Highlighting}
\end{Shaded}

\begin{verbatim}
##  [1] -0.1296364813  0.1811323957 -0.5299060523  0.0124908790  0.2713109474
##  [6]  3.0055899345  0.7016799736 -0.0006567022  1.0727559368  2.2152931318
\end{verbatim}

\begin{Shaded}
\begin{Highlighting}[]
\NormalTok{Hinv }\OtherTok{\textless{}{-}} \FunctionTok{solve}\NormalTok{(}\FunctionTok{HHH}\NormalTok{(theta\_hat))}
\NormalTok{theta\_tilde }\OtherTok{\textless{}{-}}\NormalTok{ theta\_hat }\SpecialCharTok{+}\NormalTok{ Hinv }\SpecialCharTok{\%*\%}\NormalTok{ A}
\end{Highlighting}
\end{Shaded}

\begin{tabular}[t]{lrrr}
\toprule
coef & theta\_hat & theta\_true & theta\_tilde\\
\midrule
eta1=\textasciitilde{}y1 & 0.70 & 0.80 & 0.10\\
eta1=\textasciitilde{}y2 & 0.70 & 0.70 & 2.15\\
eta1=\textasciitilde{}y3 & 0.45 & 0.47 & -1.20\\
eta1=\textasciitilde{}y4 & 0.49 & 0.38 & 0.57\\
eta1=\textasciitilde{}y5 & 0.42 & 0.34 & 1.44\\
\addlinespace
y1|t1 & -1.52 & -1.43 & 1.38\\
y2|t1 & -0.55 & -0.55 & -0.06\\
y3|t1 & -0.17 & -0.13 & -0.04\\
y4|t1 & -0.68 & -0.72 & -0.05\\
y5|t1 & -1.19 & -1.13 & 0.41\\
\bottomrule
\end{tabular}

\hypertarget{references}{%
\section*{References}\label{references}}
\addcontentsline{toc}{section}{References}

\hypertarget{refs}{}
\begin{CSLReferences}{1}{0}
\leavevmode\vadjust pre{\hypertarget{ref-jamil2023pairwise}{}}%
Jamil, Haziq, Irini Moustaki, and Chris Skinner. 2023. {``Pairwise
Likelihood Estimation and Limited Information Goodness-of-Fit Test
Statistics for Binary Factor Analysis Models Under Complex Survey
Sampling.''} \emph{arXiv Preprint arXiv:2311.02543}.

\end{CSLReferences}

\end{document}
